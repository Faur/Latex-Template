\section{Demo} \label{sec:demo}

\large{The demo is intended to be a copy paste repository, as well as a standards reference. The demo, as well as the \textit{report} class are works in progress, feedback and suggestions are welcome.}

\subsection{General writing}
\info{Short introductions, quotes, and similar text that one wishes to highlight can be done so with the $\backslash$info command}

\tdo{How to reference: $\backslash$cref, $\backslash$pageref, $\backslash$cite, $\backslash$citep, $\backslash$citet}


Label abreviations in use:
\begin{enumerate}[leftmargin=1.2 cm, label=Label \arabic*:] 
	%\roman), \Roman), \arabic*:
\itemsep1pt \parskip0pt \parsep0pt \label{list:labels}
% \setcounter{enumi}{4}
	\item eq  (equation)
	\item fig (figure)
	\item sec (section)
	\item tab (table)
	\item list 
	\item code
	\item misc (miscellaneous)
\end{enumerate} %itemize, enumerate 

\textbf{Normal text looks like this:}\\
\lipsum[2-3]


\subsection{Figures}
%%% FIGURE
\begin{figure}[htbp!] %H: Force picture right here
	\centering
	\includegraphics[width=0.8\textwidth]{./appendix/xdemo_monkey.png}
	\caption{\tdr{Add caption}}
	\label{fig:monkey}
\end{figure}

\FloatBarrier %Prevents floats (figures, tables, etc) to go past this point

\subsection{Equations}
Multi line, single label
\begin{align} \label{eq:simple}
\begin{split}
A \rightarrow B \\
B \rightarrow C \\
A \rightarrow C
\end{split}
\end{align}

Multi line, multi label
\begin{align} 
A \rightarrow B \label{eq:line1}\\
B \rightarrow C \label{eq:line2}\\
A \rightarrow C \label{eq:line3}
\end{align}

Pretty normal and square parenthesis can be made with $\backslash$( and $\backslash$[ respectively:
\begin{align} \label{eq:para}
\begin{split}
\rho (x) = \sum_{i=0}^{n} \[{ \({\frac{f(x)}{i} - y}^2}
\end{split}
\end{align}


\subsection{Inserting code}
Code can be done in-line with \textit{$\backslash$lstinline[language = XXX]}: \lstinline[language = C++]@for i=1:10 {x_1=x_1+i}@. %%\end{lstlisting}
The default language is C++. You can use any symbol that isn't in the code snippet to boarder it. \textit{lstinline} fucks up syntax highlighting in TeX-Maker, but it can be fixed with a comment: $\backslash$end\{lstlisting\}. %\end{lstlisting}
The default can be accessed through $\backslash$c (using \$ as boundaries).

Larger code segemenst can be written in a box of its own:\\
\begin{minipage}[c]{0.8\textwidth}  %Prevents page break
\centering
\begin{lstlisting}[
	language=C++, 
	firstnumber=7,
	frame=single,
	caption=Code segment in TeX]
// debug("ptr = %p, ptr->node = %p.", ptr, ptr->node);
if(stack_ptr->node == NULL) {
	free(stack_ptr);
	printf("This is not a string\n");
	return 0;
} else{
	sl_destroy_all(stack_ptr->node);
}
\end{lstlisting}
\label{code:inline}
\end{minipage}

It is also possible to load code from a file, as \\
\begin{minipage}[c]{0.8\textwidth}  %Prevents page break
\lstinputlisting[
	language=C++, 
%	firstline  =10,
%	lastline   =13,
	firstnumber=10,
	frame=single,
	caption=Code segment from file,]
	{./appendix/xdemo_code.txt} 
	\label{code:file}
\end{minipage}


\subsection{Conditionals (if, else)}
\newtoggle{myVariable} %default is false
\togglefalse{myVariable}
\toggletrue{myVariable}

Conditionals are easy to use:\\
\iftoggle{myVariable}{ % === IF ===
True
}{ % === ELSE ===
False
} % === END IF ===

